% Options for packages loaded elsewhere
\PassOptionsToPackage{unicode}{hyperref}
\PassOptionsToPackage{hyphens}{url}
%
\documentclass[
  12pt,
]{article}
\usepackage{amsmath,amssymb}
\usepackage{lmodern}
\usepackage{iftex}
\ifPDFTeX
  \usepackage[T1]{fontenc}
  \usepackage[utf8]{inputenc}
  \usepackage{textcomp} % provide euro and other symbols
\else % if luatex or xetex
  \usepackage{unicode-math}
  \defaultfontfeatures{Scale=MatchLowercase}
  \defaultfontfeatures[\rmfamily]{Ligatures=TeX,Scale=1}
\fi
% Use upquote if available, for straight quotes in verbatim environments
\IfFileExists{upquote.sty}{\usepackage{upquote}}{}
\IfFileExists{microtype.sty}{% use microtype if available
  \usepackage[]{microtype}
  \UseMicrotypeSet[protrusion]{basicmath} % disable protrusion for tt fonts
}{}
\makeatletter
\@ifundefined{KOMAClassName}{% if non-KOMA class
  \IfFileExists{parskip.sty}{%
    \usepackage{parskip}
  }{% else
    \setlength{\parindent}{0pt}
    \setlength{\parskip}{6pt plus 2pt minus 1pt}}
}{% if KOMA class
  \KOMAoptions{parskip=half}}
\makeatother
\usepackage{xcolor}
\IfFileExists{xurl.sty}{\usepackage{xurl}}{} % add URL line breaks if available
\IfFileExists{bookmark.sty}{\usepackage{bookmark}}{\usepackage{hyperref}}
\hypersetup{
  pdftitle={Prueba de evaluación R},
  pdfauthor={Ivonne Yáñez Mendoza},
  hidelinks,
  pdfcreator={LaTeX via pandoc}}
\urlstyle{same} % disable monospaced font for URLs
\usepackage[margin=1in]{geometry}
\usepackage{color}
\usepackage{fancyvrb}
\newcommand{\VerbBar}{|}
\newcommand{\VERB}{\Verb[commandchars=\\\{\}]}
\DefineVerbatimEnvironment{Highlighting}{Verbatim}{commandchars=\\\{\}}
% Add ',fontsize=\small' for more characters per line
\usepackage{framed}
\definecolor{shadecolor}{RGB}{248,248,248}
\newenvironment{Shaded}{\begin{snugshade}}{\end{snugshade}}
\newcommand{\AlertTok}[1]{\textcolor[rgb]{0.94,0.16,0.16}{#1}}
\newcommand{\AnnotationTok}[1]{\textcolor[rgb]{0.56,0.35,0.01}{\textbf{\textit{#1}}}}
\newcommand{\AttributeTok}[1]{\textcolor[rgb]{0.77,0.63,0.00}{#1}}
\newcommand{\BaseNTok}[1]{\textcolor[rgb]{0.00,0.00,0.81}{#1}}
\newcommand{\BuiltInTok}[1]{#1}
\newcommand{\CharTok}[1]{\textcolor[rgb]{0.31,0.60,0.02}{#1}}
\newcommand{\CommentTok}[1]{\textcolor[rgb]{0.56,0.35,0.01}{\textit{#1}}}
\newcommand{\CommentVarTok}[1]{\textcolor[rgb]{0.56,0.35,0.01}{\textbf{\textit{#1}}}}
\newcommand{\ConstantTok}[1]{\textcolor[rgb]{0.00,0.00,0.00}{#1}}
\newcommand{\ControlFlowTok}[1]{\textcolor[rgb]{0.13,0.29,0.53}{\textbf{#1}}}
\newcommand{\DataTypeTok}[1]{\textcolor[rgb]{0.13,0.29,0.53}{#1}}
\newcommand{\DecValTok}[1]{\textcolor[rgb]{0.00,0.00,0.81}{#1}}
\newcommand{\DocumentationTok}[1]{\textcolor[rgb]{0.56,0.35,0.01}{\textbf{\textit{#1}}}}
\newcommand{\ErrorTok}[1]{\textcolor[rgb]{0.64,0.00,0.00}{\textbf{#1}}}
\newcommand{\ExtensionTok}[1]{#1}
\newcommand{\FloatTok}[1]{\textcolor[rgb]{0.00,0.00,0.81}{#1}}
\newcommand{\FunctionTok}[1]{\textcolor[rgb]{0.00,0.00,0.00}{#1}}
\newcommand{\ImportTok}[1]{#1}
\newcommand{\InformationTok}[1]{\textcolor[rgb]{0.56,0.35,0.01}{\textbf{\textit{#1}}}}
\newcommand{\KeywordTok}[1]{\textcolor[rgb]{0.13,0.29,0.53}{\textbf{#1}}}
\newcommand{\NormalTok}[1]{#1}
\newcommand{\OperatorTok}[1]{\textcolor[rgb]{0.81,0.36,0.00}{\textbf{#1}}}
\newcommand{\OtherTok}[1]{\textcolor[rgb]{0.56,0.35,0.01}{#1}}
\newcommand{\PreprocessorTok}[1]{\textcolor[rgb]{0.56,0.35,0.01}{\textit{#1}}}
\newcommand{\RegionMarkerTok}[1]{#1}
\newcommand{\SpecialCharTok}[1]{\textcolor[rgb]{0.00,0.00,0.00}{#1}}
\newcommand{\SpecialStringTok}[1]{\textcolor[rgb]{0.31,0.60,0.02}{#1}}
\newcommand{\StringTok}[1]{\textcolor[rgb]{0.31,0.60,0.02}{#1}}
\newcommand{\VariableTok}[1]{\textcolor[rgb]{0.00,0.00,0.00}{#1}}
\newcommand{\VerbatimStringTok}[1]{\textcolor[rgb]{0.31,0.60,0.02}{#1}}
\newcommand{\WarningTok}[1]{\textcolor[rgb]{0.56,0.35,0.01}{\textbf{\textit{#1}}}}
\usepackage{graphicx}
\makeatletter
\def\maxwidth{\ifdim\Gin@nat@width>\linewidth\linewidth\else\Gin@nat@width\fi}
\def\maxheight{\ifdim\Gin@nat@height>\textheight\textheight\else\Gin@nat@height\fi}
\makeatother
% Scale images if necessary, so that they will not overflow the page
% margins by default, and it is still possible to overwrite the defaults
% using explicit options in \includegraphics[width, height, ...]{}
\setkeys{Gin}{width=\maxwidth,height=\maxheight,keepaspectratio}
% Set default figure placement to htbp
\makeatletter
\def\fps@figure{htbp}
\makeatother
\setlength{\emergencystretch}{3em} % prevent overfull lines
\providecommand{\tightlist}{%
  \setlength{\itemsep}{0pt}\setlength{\parskip}{0pt}}
\setcounter{secnumdepth}{-\maxdimen} % remove section numbering
\usepackage{booktabs}
\usepackage{longtable}
\usepackage{array}
\usepackage{multirow}
\usepackage{wrapfig}
\usepackage{float}
\usepackage{colortbl}
\usepackage{pdflscape}
\usepackage{tabu}
\usepackage{threeparttable}
\usepackage{threeparttablex}
\usepackage[normalem]{ulem}
\usepackage{makecell}
\usepackage{xcolor}
\ifLuaTeX
  \usepackage{selnolig}  % disable illegal ligatures
\fi

\title{Prueba de evaluación R}
\author{Ivonne Yáñez Mendoza}
\date{}

\begin{document}
\maketitle

\hypertarget{muxf3dulo-programacion-r}{%
\subparagraph{Módulo: Programacion R}\label{muxf3dulo-programacion-r}}

\hypertarget{profesor-jose-luis-brita-paja}{%
\subparagraph{Profesor: Jose Luis
Brita-Paja}\label{profesor-jose-luis-brita-paja}}

\hypertarget{master-en-big-data-business-analytics-universidad-complutense-de-madrid}{%
\subparagraph{Master en big data \& business analytics, Universidad
Complutense de
Madrid}\label{master-en-big-data-business-analytics-universidad-complutense-de-madrid}}

\hypertarget{de-abril-de-2022}{%
\subparagraph{26 de abril de 2022}\label{de-abril-de-2022}}

\begin{center}\rule{0.5\linewidth}{0.5pt}\end{center}

\textbf{Pregunta 1}

\begin{enumerate}
\def\labelenumi{\alph{enumi})}
\tightlist
\item
  Generar todos los números que entran en el sorteo de la ONCE y
  mostrarlos con los cuatro dígitos.
\end{enumerate}

\begin{Shaded}
\begin{Highlighting}[]
\CommentTok{\# Se genera el dato para trabajar}
\NormalTok{numeros }\OtherTok{\textless{}{-}} \FunctionTok{c}\NormalTok{(}\FunctionTok{formatC}\NormalTok{(}\DecValTok{0000}\SpecialCharTok{:}\DecValTok{9999}\NormalTok{, }\AttributeTok{width =} \DecValTok{4}\NormalTok{, }\AttributeTok{flag =} \StringTok{\textquotesingle{}0\textquotesingle{}}\NormalTok{ ))}

\CommentTok{\# El resultado }
\NormalTok{resultado }\OtherTok{\textless{}{-}} \FunctionTok{sapply}\NormalTok{(numeros, }\ControlFlowTok{function}\NormalTok{(x) }
          \FunctionTok{sum}\NormalTok{( }\FunctionTok{as.numeric}\NormalTok{(}\FunctionTok{unlist}\NormalTok{(}\FunctionTok{strsplit}\NormalTok{(}\FunctionTok{as.character}\NormalTok{(x), }\AttributeTok{split=}\StringTok{""}\NormalTok{))) ))}

\NormalTok{ordenado }\OtherTok{\textless{}{-}} \FunctionTok{sort.int}\NormalTok{(resultado, }\AttributeTok{decreasing =} \ConstantTok{TRUE}\NormalTok{)}
\end{Highlighting}
\end{Shaded}

\begin{enumerate}
\def\labelenumi{\alph{enumi})}
\setcounter{enumi}{1}
\tightlist
\item
  ¿Cuál es la suma de los números de un boleto que más se repite?
\end{enumerate}

\begin{Shaded}
\begin{Highlighting}[]
\NormalTok{mas\_repetido }\OtherTok{\textless{}{-}} \FunctionTok{tail}\NormalTok{(}\FunctionTok{names}\NormalTok{(}\FunctionTok{sort}\NormalTok{(}\FunctionTok{table}\NormalTok{(resultado))), }\DecValTok{1}\NormalTok{)}
\FunctionTok{sprintf}\NormalTok{(}\StringTok{"La suma de los numeros de un boleto que mas se repite es \%s. "}\NormalTok{, }
\NormalTok{        mas\_repetido)}
\end{Highlighting}
\end{Shaded}

\begin{verbatim}
## [1] "La suma de los numeros de un boleto que mas se repite es 18. "
\end{verbatim}

\textbf{Pregunta 2}

\begin{enumerate}
\def\labelenumi{\alph{enumi})}
\tightlist
\item
  Leer los archivos ``datos\_provincias.csv'', ``CodProv.txt*
  y''CodCCAA.dat ``. Añade el código de la comunidad autónoma al
  fichero''datos\_provincias.csv'' (no manualmente).
\end{enumerate}

\begin{Shaded}
\begin{Highlighting}[]
\FunctionTok{library}\NormalTok{(kableExtra)}
\CommentTok{\# Paso 1 leer los archivos}

\NormalTok{cod\_ccaa }\OtherTok{\textless{}{-}} \FunctionTok{read.csv}\NormalTok{(}\StringTok{"CodCCAA.csv"}\NormalTok{)}
\NormalTok{datos\_provincias }\OtherTok{\textless{}{-}} \FunctionTok{data.frame}\NormalTok{(}\FunctionTok{read.csv}\NormalTok{(}\StringTok{"datos\_provincias.csv"}\NormalTok{))}
\NormalTok{cod\_provincias }\OtherTok{\textless{}{-}} \FunctionTok{data.frame}\NormalTok{(}\FunctionTok{read.table}\NormalTok{(}\AttributeTok{file =} \StringTok{"CodProv.txt"}\NormalTok{, }\AttributeTok{sep=}\StringTok{","}\NormalTok{, }
                  \AttributeTok{header =} \ConstantTok{TRUE}\NormalTok{, }\AttributeTok{row.names =} \ConstantTok{NULL}\NormalTok{))}

\CommentTok{\# Paso 2 se elimina la ES de la columna Codigo del data frame cod\_provincias}
\NormalTok{cod\_provincias}\SpecialCharTok{$}\NormalTok{Código }\OtherTok{\textless{}{-}} \FunctionTok{sub}\NormalTok{(}\StringTok{"\^{}(}\SpecialCharTok{\textbackslash{}\textbackslash{}}\StringTok{w+){-}"}\NormalTok{, }\StringTok{""}\NormalTok{, cod\_provincias}\SpecialCharTok{$}\NormalTok{Código)}

\CommentTok{\# Con merge se une tanto el data frame cod\_provincias como datos provincias }
\CommentTok{\# para traer el codigo de la comunidad autonoma que corresponde}

\NormalTok{base\_graficos }\OtherTok{\textless{}{-}} \FunctionTok{merge}\NormalTok{(cod\_provincias, datos\_provincias, }\AttributeTok{by.x =} \StringTok{"Código"}\NormalTok{, }\AttributeTok{by.y =} \StringTok{"provincia\_iso"}\NormalTok{, }\AttributeTok{all.y =} \ConstantTok{TRUE}\NormalTok{)}
\end{Highlighting}
\end{Shaded}

\begin{enumerate}
\def\labelenumi{\alph{enumi})}
\setcounter{enumi}{1}
\tightlist
\item
  Selecciona los datos de la comunidad autónoma que te corresponda. Para
  saber cuál es tu comunidad autónoma realiza la siguiente operación
\end{enumerate}

\begin{Shaded}
\begin{Highlighting}[]
\CommentTok{\# DNI o Pasaporte mod 17 por ejemplo (12345678 \%\% 17 = 6 → Castilla y León)}

\CommentTok{\# Paso 2 seleccionar los datos de la comunidad autonoma que me corresponde.}

\NormalTok{dni }\OtherTok{\textless{}{-}} \DecValTok{169728542}
\NormalTok{dni\_provincias }\OtherTok{\textless{}{-}}\NormalTok{ dni }\SpecialCharTok{\%\%} \DecValTok{17}

\CommentTok{\# Según este cálculo la provincia asignada es PV, pais vasco, se filtra segun }
\CommentTok{\# el código de la comunidad autónoma"}
\NormalTok{viscaya }\OtherTok{\textless{}{-}} \FunctionTok{subset}\NormalTok{(base\_graficos, Comunidad.autónoma }\SpecialCharTok{==} \StringTok{"PV"}\NormalTok{)}
\end{Highlighting}
\end{Shaded}

\textbf{Gráficos}

\begin{enumerate}
\def\labelenumi{\alph{enumi})}
\setcounter{enumi}{2}
\tightlist
\item
  Realizar un gráfico que muestre adecuadamente la evolución de los
  casos nuevos. Justifica el gráfico elegido. Nota: Elegí este tipo de
  gráfico pues muestra de una forma visual el aumento y decremento de
  los picos de casos Covid para el periodo señalado.
\end{enumerate}

\begin{Shaded}
\begin{Highlighting}[]
\NormalTok{datos }\OtherTok{\textless{}{-}}\NormalTok{ viscaya}

\CommentTok{\# Con los datos de Viscaya se genera el gráfico}
\NormalTok{datos}\SpecialCharTok{$}\NormalTok{fecha }\OtherTok{\textless{}{-}} \FunctionTok{as.Date}\NormalTok{(datos}\SpecialCharTok{$}\NormalTok{fecha)}
\NormalTok{datos }\OtherTok{\textless{}{-}}\NormalTok{ datos[}\FunctionTok{order}\NormalTok{(datos}\SpecialCharTok{$}\NormalTok{fecha),]}


\NormalTok{x }\OtherTok{\textless{}{-}}\NormalTok{ datos}\SpecialCharTok{$}\NormalTok{fecha}
\NormalTok{y }\OtherTok{\textless{}{-}}\NormalTok{ datos}\SpecialCharTok{$}\NormalTok{num\_casos}


\NormalTok{g }\OtherTok{\textless{}{-}} \FunctionTok{aggregate}\NormalTok{(y, }\AttributeTok{by=}\FunctionTok{list}\NormalTok{(}\AttributeTok{casos=}\NormalTok{x), }\AttributeTok{FUN=}\NormalTok{sum)}

\StringTok{"grafico 1"}
\end{Highlighting}
\end{Shaded}

\begin{verbatim}
## [1] "grafico 1"
\end{verbatim}

\begin{Shaded}
\begin{Highlighting}[]
\FunctionTok{plot}\NormalTok{(g, }\AttributeTok{type =}\StringTok{"l"}\NormalTok{, }\AttributeTok{xlab =}\StringTok{"Meses 2020"}\NormalTok{, }\AttributeTok{ylab =}\StringTok{"Numero casos"}\NormalTok{, }\AttributeTok{main =}\StringTok{"Evolución Casos Diarios Comunidad Autonoma Viscaya"}\NormalTok{, }\AttributeTok{col =} \StringTok{"blue"}\NormalTok{, }\AttributeTok{font =} \DecValTok{2}\NormalTok{)}
\FunctionTok{polygon}\NormalTok{(g, }\AttributeTok{col =} \StringTok{"lightblue"}\NormalTok{)}
\FunctionTok{grid}\NormalTok{()}
\FunctionTok{axis}\NormalTok{(}\AttributeTok{side=} \DecValTok{2}\NormalTok{, }\AttributeTok{font=}\DecValTok{2}\NormalTok{)}
\end{Highlighting}
\end{Shaded}

\includegraphics{iymprueba_files/figure-latex/unnamed-chunk-5-1.pdf}

\begin{enumerate}
\def\labelenumi{\alph{enumi})}
\setcounter{enumi}{3}
\tightlist
\item
  Presenta en único gráfico la evolución de las distintas variables
  (columnas) por medio de un gráfico de líneas múltiples. Utiliza
  diferentes colores y añade una leyenda muestre el origen de cada
  línea.
\end{enumerate}

\begin{Shaded}
\begin{Highlighting}[]
\StringTok{"grafico 2"}
\end{Highlighting}
\end{Shaded}

\begin{verbatim}
## [1] "grafico 2"
\end{verbatim}

\begin{Shaded}
\begin{Highlighting}[]
\FunctionTok{plot}\NormalTok{(g, }\AttributeTok{type =} \StringTok{"s"}\NormalTok{, }
     \AttributeTok{xtat =} \StringTok{\textquotesingle{}n\textquotesingle{}}\NormalTok{, }\AttributeTok{ylab =} \StringTok{\textquotesingle{}Numero casos\textquotesingle{}}\NormalTok{, }\AttributeTok{xlab =} \StringTok{"Meses 2020"}\NormalTok{,}
     \AttributeTok{main =} \StringTok{"Evolucion de casos COVID segun tipo de caso"}\NormalTok{, }\AttributeTok{beside =} \ConstantTok{TRUE}\NormalTok{)}


\NormalTok{a }\OtherTok{\textless{}{-}} \FunctionTok{aggregate}\NormalTok{(datos}\SpecialCharTok{$}\NormalTok{num\_casos\_prueba\_pcr, }\AttributeTok{by=}\FunctionTok{list}\NormalTok{(}\AttributeTok{casos =}\NormalTok{ datos}\SpecialCharTok{$}\NormalTok{fecha), }\AttributeTok{FUN =} \StringTok{"sum"}\NormalTok{)}
\NormalTok{b }\OtherTok{\textless{}{-}} \FunctionTok{aggregate}\NormalTok{(datos}\SpecialCharTok{$}\NormalTok{num\_casos\_prueba\_otras, }\AttributeTok{by=} \FunctionTok{list}\NormalTok{(}\AttributeTok{casos =}\NormalTok{ datos}\SpecialCharTok{$}\NormalTok{fecha), }\AttributeTok{FUN =} \StringTok{\textquotesingle{}sum\textquotesingle{}}\NormalTok{)}
\NormalTok{c }\OtherTok{\textless{}{-}} \FunctionTok{aggregate}\NormalTok{(datos}\SpecialCharTok{$}\NormalTok{num\_casos\_prueba\_test\_ac, }\AttributeTok{by =} \FunctionTok{list}\NormalTok{(}\AttributeTok{casos =}\NormalTok{ datos}\SpecialCharTok{$}\NormalTok{fecha), }\AttributeTok{FUN =} \StringTok{\textquotesingle{}sum\textquotesingle{}}\NormalTok{)}
\NormalTok{d }\OtherTok{\textless{}{-}} \FunctionTok{aggregate}\NormalTok{(datos}\SpecialCharTok{$}\NormalTok{num\_casos\_prueba\_desconocida, }\AttributeTok{by =} \FunctionTok{list}\NormalTok{(}\AttributeTok{casos =}\NormalTok{ datos}\SpecialCharTok{$}\NormalTok{fecha), }\AttributeTok{FUN =} \StringTok{\textquotesingle{}sum\textquotesingle{}}\NormalTok{)}
\FunctionTok{lines}\NormalTok{(a, }\AttributeTok{col=} \StringTok{\textquotesingle{}red\textquotesingle{}}\NormalTok{)}
\FunctionTok{lines}\NormalTok{(b, }\AttributeTok{col =} \StringTok{\textquotesingle{}blue\textquotesingle{}}\NormalTok{)}
\FunctionTok{lines}\NormalTok{(c, }\AttributeTok{col =} \StringTok{\textquotesingle{}green\textquotesingle{}}\NormalTok{)}
\FunctionTok{lines}\NormalTok{(d,}\AttributeTok{col =} \StringTok{\textquotesingle{}purple\textquotesingle{}}\NormalTok{)}

\FunctionTok{legend}\NormalTok{(}\AttributeTok{x =} \StringTok{"topleft"}\NormalTok{,         }\CommentTok{\# Posición}
       \AttributeTok{title =} \StringTok{"Tipo casos"}\NormalTok{,}
       \AttributeTok{legend =} \FunctionTok{c}\NormalTok{(}\StringTok{"casos"}\NormalTok{, }\StringTok{"casos pcr"}\NormalTok{, }\StringTok{"casos otras"}\NormalTok{, }\StringTok{"casos ac"}\NormalTok{, }\StringTok{"desconocido"}\NormalTok{), }\CommentTok{\# Textos de la leyenda}
       \AttributeTok{lty =} \FunctionTok{c}\NormalTok{(}\DecValTok{1}\NormalTok{),          }\CommentTok{\# Tipo de líneas}
       \AttributeTok{col =} \FunctionTok{c}\NormalTok{(}\DecValTok{1}\NormalTok{, }\StringTok{\textquotesingle{}red\textquotesingle{}}\NormalTok{, }\StringTok{\textquotesingle{}blue\textquotesingle{}}\NormalTok{, }\StringTok{\textquotesingle{}green\textquotesingle{}}\NormalTok{, }\StringTok{\textquotesingle{}purple\textquotesingle{}}\NormalTok{),          }\CommentTok{\# Colores de las líneas}
       \AttributeTok{lwd =} \DecValTok{2}\NormalTok{)   }
\FunctionTok{grid}\NormalTok{()}
\end{Highlighting}
\end{Shaded}

\includegraphics{iymprueba_files/figure-latex/unnamed-chunk-6-1.pdf}

\textbf{Pregunta 3}

\begin{enumerate}
\def\labelenumi{\alph{enumi})}
\tightlist
\item
  Crear dos data frame de nombres, uno de nombre articulo leyendo la
  información de ``articulo.xlsx'' y el otro de nombre descuento que
  guarde la información de ``descuento\_aplicar.txt''.
\end{enumerate}

\begin{Shaded}
\begin{Highlighting}[]
\FunctionTok{library}\NormalTok{(}\StringTok{"xlsx"}\NormalTok{) }\CommentTok{\#Para leer archivos excel}

\CommentTok{\# Paso 1 leer los datos del archivo excel y del archivo txt"}
\NormalTok{tabla\_excel }\OtherTok{\textless{}{-}} \FunctionTok{read.xlsx}\NormalTok{(}\StringTok{"articulo.xlsx"}\NormalTok{, }\AttributeTok{sheetIndex =} \DecValTok{1}\NormalTok{)}
\NormalTok{descuento }\OtherTok{\textless{}{-}} \FunctionTok{read.delim}\NormalTok{(}\StringTok{"descuento\_aplicar.txt"}\NormalTok{)}


\CommentTok{\# Paso 2 transformar la tabla excel y el archivo de texto a un data frame}
\NormalTok{articulos }\OtherTok{\textless{}{-}} \FunctionTok{data.frame}\NormalTok{(tabla\_excel)}
\NormalTok{descuentos }\OtherTok{\textless{}{-}} \FunctionTok{data.frame}\NormalTok{(descuento)}

\CommentTok{\# Cálculo de ingreso bruto}
\NormalTok{articulos}\SpecialCharTok{$}\NormalTok{ingreso\_bruto }\OtherTok{\textless{}{-}}\NormalTok{ articulos}\SpecialCharTok{$}\NormalTok{PVP }\SpecialCharTok{*}\NormalTok{ articulos}\SpecialCharTok{$}\NormalTok{CANTIDAD }

\NormalTok{articulos }\OtherTok{\textless{}{-}}\NormalTok{ articulos[}\FunctionTok{order}\NormalTok{(articulos}\SpecialCharTok{$}\NormalTok{ingreso\_bruto, }\AttributeTok{decreasing =} \ConstantTok{TRUE}\NormalTok{),]}
\FunctionTok{print}\NormalTok{(}\FunctionTok{head}\NormalTok{(articulos))}
\end{Highlighting}
\end{Shaded}

\begin{verbatim}
##    PRODUCTO   PVP CANTIDAD ingreso_bruto
## 5         5 10800     4050      43740000
## 29       29  6600     5800      38280000
## 1         1  5000     6000      30000000
## 36       37  9700     3030      29391000
## 28       28 11500     2000      23000000
## 15       15  5400     4000      21600000
\end{verbatim}

\begin{enumerate}
\def\labelenumi{\alph{enumi})}
\setcounter{enumi}{1}
\tightlist
\item
  Crear una variable llamada tipo que clasifique los artículos en los
  tipos A, B y C en la forma indicada y añadirlo al data frame artículo
  y calcula, en una única sentencia, el total de ingresos brutos por
  cada tipo de artículo.
\end{enumerate}

\begin{Shaded}
\begin{Highlighting}[]
\CommentTok{\# Calcula deciles para utilizarlos en el cálculo del tipo de descuento}

\NormalTok{deciles }\OtherTok{\textless{}{-}} \FunctionTok{quantile}\NormalTok{(articulos}\SpecialCharTok{$}\NormalTok{ingreso\_bruto,}\AttributeTok{probs =} \FunctionTok{seq}\NormalTok{(.}\DecValTok{1}\NormalTok{, .}\DecValTok{9}\NormalTok{, }\AttributeTok{by =}\NormalTok{ .}\DecValTok{1}\NormalTok{))}

\CommentTok{\# Asigna tramo de descuento según ingreso bruto y decil}
\NormalTok{articulos}\SpecialCharTok{$}\NormalTok{tipo }\OtherTok{\textless{}{-}} \FunctionTok{with}\NormalTok{(articulos, }\FunctionTok{ifelse}\NormalTok{(ingreso\_bruto }\SpecialCharTok{\textgreater{}=} \DecValTok{6108000}\NormalTok{, }\StringTok{\textquotesingle{}A\textquotesingle{}}\NormalTok{,}
                                  \FunctionTok{ifelse}\NormalTok{(ingreso\_bruto }\SpecialCharTok{\textgreater{}} \DecValTok{1797000} \SpecialCharTok{\&}\NormalTok{ ingreso\_bruto }\SpecialCharTok{\textless{}} \DecValTok{6108000}\NormalTok{, }\StringTok{"B"}\NormalTok{,}
                                  \FunctionTok{ifelse}\NormalTok{(ingreso\_bruto }\SpecialCharTok{\textless{}=} \DecValTok{1797000}\NormalTok{, }\StringTok{"C"}\NormalTok{, }\DecValTok{0}\NormalTok{))))}

\CommentTok{\# Ordena el data frame de mayor a menor ingreso bruto}
\NormalTok{articulos }\OtherTok{\textless{}{-}}\NormalTok{ articulos[}\FunctionTok{order}\NormalTok{(articulos}\SpecialCharTok{$}\NormalTok{ingreso\_bruto, }\AttributeTok{decreasing =} \ConstantTok{TRUE}\NormalTok{), ]}
\FunctionTok{print}\NormalTok{(}\FunctionTok{head}\NormalTok{(articulos))}
\end{Highlighting}
\end{Shaded}

\begin{verbatim}
##    PRODUCTO   PVP CANTIDAD ingreso_bruto tipo
## 5         5 10800     4050      43740000    A
## 29       29  6600     5800      38280000    A
## 1         1  5000     6000      30000000    A
## 36       37  9700     3030      29391000    A
## 28       28 11500     2000      23000000    A
## 15       15  5400     4000      21600000    A
\end{verbatim}

\begin{enumerate}
\def\labelenumi{\alph{enumi})}
\setcounter{enumi}{2}
\tightlist
\item
  Unir los dos data frame, articulo y descuento, de forma adecuada, para
  crear el data frame clientes. Calcular la variable nuevo\_pvp = pvp --
  cantidad a descontar. Suponiendo que el volumen de ventas se mantiene
  constante dar una estimación del porcentaje de decremento en los
  ingresos brutos al aplicar los descuentos.
\end{enumerate}

\begin{Shaded}
\begin{Highlighting}[]
\CommentTok{\# Se unen los data frame articulos y descuento para crear el df clientes en el paso posterior}
\NormalTok{articulos2 }\OtherTok{\textless{}{-}}\NormalTok{ articulos}
\NormalTok{articulos3 }\OtherTok{\textless{}{-}} \FunctionTok{merge}\NormalTok{(articulos, descuento, }\AttributeTok{by.x =} \StringTok{"tipo"}\NormalTok{, }\AttributeTok{by.y =} \StringTok{"tipo"}\NormalTok{, }\AttributeTok{all.x =} \ConstantTok{TRUE}\NormalTok{)}


\CommentTok{\# Crea el data frame clientes"}

\NormalTok{clientes }\OtherTok{\textless{}{-}}\NormalTok{ articulos3}
\FunctionTok{print}\NormalTok{(}\FunctionTok{head}\NormalTok{(clientes))}
\end{Highlighting}
\end{Shaded}

\begin{verbatim}
##   tipo PRODUCTO   PVP CANTIDAD ingreso_bruto descuento
## 1    A        5 10800     4050      43740000        10
## 2    A       29  6600     5800      38280000        10
## 3    A        1  5000     6000      30000000        10
## 4    A       37  9700     3030      29391000        10
## 5    A       28 11500     2000      23000000        10
## 6    A       15  5400     4000      21600000        10
\end{verbatim}

\begin{Shaded}
\begin{Highlighting}[]
\CommentTok{\# Calcular la variable nuevo\_pvp con el \% de descuento correspondiente segun tramo"}
\NormalTok{clientes}\SpecialCharTok{$}\NormalTok{nuevo\_pvp }\OtherTok{\textless{}{-}} \FunctionTok{with}\NormalTok{(clientes,}\FunctionTok{ifelse}\NormalTok{(tipo }\SpecialCharTok{==} \StringTok{"A"}\NormalTok{, clientes}\SpecialCharTok{$}\NormalTok{PVP }\SpecialCharTok{*} \FloatTok{0.9}\NormalTok{,}
                                             \FunctionTok{ifelse}\NormalTok{(tipo }\SpecialCharTok{==} \StringTok{"B"}\NormalTok{, clientes}\SpecialCharTok{$}\NormalTok{PVP }\SpecialCharTok{*} \FloatTok{0.85}\NormalTok{, }
                                            \FunctionTok{ifelse}\NormalTok{(tipo }\SpecialCharTok{==} \StringTok{"C"}\NormalTok{, clientes}\SpecialCharTok{$}\NormalTok{PVP }\SpecialCharTok{*} \FloatTok{0.8}\NormalTok{, }\DecValTok{0}\NormalTok{))))}

\CommentTok{\# Crea una nueva columna donde calcula el ingreso bruto por artículo con los valores actualizados"}
\NormalTok{clientes}\SpecialCharTok{$}\NormalTok{n\_ingreso\_bruto }\OtherTok{\textless{}{-}}\NormalTok{ clientes}\SpecialCharTok{$}\NormalTok{CANTIDAD }\SpecialCharTok{*}\NormalTok{ clientes}\SpecialCharTok{$}\NormalTok{nuevo\_pvp}

\CommentTok{\# Se realiza el cálculo de decremento"}

\NormalTok{decremento }\OtherTok{\textless{}{-}} \FunctionTok{round}\NormalTok{(}\DecValTok{100} \SpecialCharTok{{-}}\NormalTok{ ((}\FunctionTok{sum}\NormalTok{(clientes}\SpecialCharTok{$}\NormalTok{n\_ingreso\_bruto) }\SpecialCharTok{/} \FunctionTok{sum}\NormalTok{(clientes}\SpecialCharTok{$}\NormalTok{ingreso\_bruto)) }\SpecialCharTok{*} \DecValTok{100}\NormalTok{ ), }\AttributeTok{digits =} \DecValTok{2}\NormalTok{)}
\FunctionTok{paste0}\NormalTok{(}\StringTok{"El porcentaje de decremento corresponde a un "}\NormalTok{, decremento, }\StringTok{"\%"}\NormalTok{, }\StringTok{"."}\NormalTok{)}
\end{Highlighting}
\end{Shaded}

\begin{verbatim}
## [1] "El porcentaje de decremento corresponde a un 11.45%."
\end{verbatim}

\textbf{Pregunta 4}

\begin{enumerate}
\def\labelenumi{\arabic{enumi}.}
\tightlist
\item
  Para la semana S7, calcule el vector (𝑓1,1−𝑓1,𝑓2,1−𝑓2,𝑓3,1−𝑓3,𝑓4,1−𝑓4)
  donde 𝑓𝑖 es la frecuencia (relativa) de la modalidad 𝑖∈\{1,2,3,4\}
  observada en la semana 𝑆7 sobre los 16 sujetos. (Sugerencia: use las
  funciones tabulate(), cbind(), t()y as.vector()). Ahora, use la
  función apply() para hacer el mismo cálculo para todas las demás
  semanas. Almacene el resultado en una matriz.
\end{enumerate}

\begin{Shaded}
\begin{Highlighting}[]
\CommentTok{\# Leer el archivo }
\NormalTok{fichero }\OtherTok{\textless{}{-}} \FunctionTok{source}\NormalTok{(}\StringTok{"matriz.R"}\NormalTok{)}
\NormalTok{fichero }\OtherTok{\textless{}{-}}\NormalTok{ fichero[[}\DecValTok{1}\NormalTok{]]}

\FunctionTok{print}\NormalTok{(fichero)}
\end{Highlighting}
\end{Shaded}

\begin{verbatim}
##       [,1] [,2] [,3] [,4] [,5] [,6] [,7] [,8]
##  [1,]    1    1    1    1    1    1    1    1
##  [2,]    2    1    1    1    1    1    1    2
##  [3,]    3    1    1    1    1    1    2    3
##  [4,]    4    1    1    1    1    1    3    4
##  [5,]    5    1    1    1    1    2    3    3
##  [6,]    6    1    1    1    1    1    1    1
##  [7,]    7    1    1    1    3    4    2    2
##  [8,]    8    1    1    1    1    1    1    1
##  [9,]    9    1    1    1    1    1    1    1
## [10,]   10    1    1    1    1    1    1    4
## [11,]   11    1    1    1    1    1    1    1
## [12,]   12    1    1    1    1    1    1    1
## [13,]   13    1    2    1    3    2    3    4
## [14,]   14    1    1    1    2    2    4    4
## [15,]   15    1    1    1    1    1    1    1
## [16,]   16    1    1    1    1    1    1    1
\end{verbatim}

\begin{Shaded}
\begin{Highlighting}[]
\CommentTok{\# El primer paso es limpiar y organizar los datos}
\CommentTok{\# La variable base tiene por objetivo contar la cantidad de ocurrencias de los niveles y rellena con 0 en caso de no hallar coincidencias}

\NormalTok{base }\OtherTok{\textless{}{-}} \FunctionTok{apply}\NormalTok{(fichero, }\DecValTok{2}\NormalTok{, }\ControlFlowTok{function}\NormalTok{(i) }\FunctionTok{table}\NormalTok{(}\FunctionTok{factor}\NormalTok{(i, }\AttributeTok{levels =} \FunctionTok{c}\NormalTok{(}\DecValTok{1}\SpecialCharTok{:}\DecValTok{4}\NormalTok{))))}

\CommentTok{\#Se elimina la columna 1 pues no se utilizará para los cálculos y la tabla esta lista para trabajar}
\NormalTok{base }\OtherTok{\textless{}{-}}\NormalTok{ base[,}\SpecialCharTok{{-}}\DecValTok{1}\NormalTok{] }
\FunctionTok{print}\NormalTok{(base)}
\end{Highlighting}
\end{Shaded}

\begin{verbatim}
##   [,1] [,2] [,3] [,4] [,5] [,6] [,7]
## 1   16   15   16   13   12   10    8
## 2    0    1    0    1    3    2    2
## 3    0    0    0    2    0    3    2
## 4    0    0    0    0    1    1    4
\end{verbatim}

\begin{Shaded}
\begin{Highlighting}[]
\CommentTok{\# En una variable se calcula el total de registros que tiene la tabla, son 16}
\CommentTok{\# Es preferible calcularla en caso que se desee reutilizar este script}
\NormalTok{registros }\OtherTok{\textless{}{-}} \FunctionTok{nrow}\NormalTok{(fichero)}
\FunctionTok{print}\NormalTok{(registros)}
\end{Highlighting}
\end{Shaded}

\begin{verbatim}
## [1] 16
\end{verbatim}

\begin{Shaded}
\begin{Highlighting}[]
\CommentTok{\#Se realiza el cálculo de frecuencias relativas para la matriz y se agregan nombres a las columnas }

\NormalTok{matriz }\OtherTok{\textless{}{-}}\NormalTok{ base }\SpecialCharTok{/}\NormalTok{ registros}
\FunctionTok{colnames}\NormalTok{(matriz) }\OtherTok{\textless{}{-}} \FunctionTok{c}\NormalTok{(}\StringTok{"s1"}\NormalTok{, }\StringTok{"s2"}\NormalTok{, }\StringTok{"s3"}\NormalTok{, }\StringTok{"s4"}\NormalTok{, }\StringTok{"s5"}\NormalTok{, }\StringTok{"s6"}\NormalTok{, }\StringTok{"s7"}\NormalTok{)}

\FunctionTok{print}\NormalTok{(matriz)}
\end{Highlighting}
\end{Shaded}

\begin{verbatim}
##   s1     s2 s3     s4     s5     s6    s7
## 1  1 0.9375  1 0.8125 0.7500 0.6250 0.500
## 2  0 0.0625  0 0.0625 0.1875 0.1250 0.125
## 3  0 0.0000  0 0.1250 0.0000 0.1875 0.125
## 4  0 0.0000  0 0.0000 0.0625 0.0625 0.250
\end{verbatim}

\begin{Shaded}
\begin{Highlighting}[]
\CommentTok{\# Tomando como base la matriz, se crea una nueva para calcular la resta del 1 a las frecuencias}

\NormalTok{matriz2 }\OtherTok{\textless{}{-}} \FunctionTok{matrix}\NormalTok{(}\FunctionTok{unlist}\NormalTok{(}\FunctionTok{lapply}\NormalTok{(}\DecValTok{1}\NormalTok{, }\StringTok{\textasciigrave{}}\AttributeTok{{-}}\StringTok{\textasciigrave{}}\NormalTok{, matriz)), }\AttributeTok{ncol =} \DecValTok{7}\NormalTok{, }\AttributeTok{nrow=}\DecValTok{4}\NormalTok{)}
\FunctionTok{row.names}\NormalTok{(matriz2) }\OtherTok{\textless{}{-}} \FunctionTok{c}\NormalTok{(}\StringTok{"1"}\NormalTok{, }\StringTok{"2"}\NormalTok{, }\StringTok{"3"}\NormalTok{, }\StringTok{"4"}\NormalTok{)}
\FunctionTok{print}\NormalTok{(matriz2)}
\end{Highlighting}
\end{Shaded}

\begin{verbatim}
##   [,1]   [,2] [,3]   [,4]   [,5]   [,6]  [,7]
## 1    0 0.0625    0 0.1875 0.2500 0.3750 0.500
## 2    1 0.9375    1 0.9375 0.8125 0.8750 0.875
## 3    1 1.0000    1 0.8750 1.0000 0.8125 0.875
## 4    1 1.0000    1 1.0000 0.9375 0.9375 0.750
\end{verbatim}

\begin{Shaded}
\begin{Highlighting}[]
\CommentTok{\# Una vez listas ambas matrices es momento de combinarlas para generar la base del gráfico}
\NormalTok{data }\OtherTok{\textless{}{-}} \FunctionTok{rbind}\NormalTok{(matriz, matriz2)}
\NormalTok{data }\OtherTok{\textless{}{-}}\NormalTok{ data[ }\FunctionTok{order}\NormalTok{(}\FunctionTok{as.numeric}\NormalTok{(}\FunctionTok{row.names}\NormalTok{(data))), ]}
\FunctionTok{print}\NormalTok{(data)}
\end{Highlighting}
\end{Shaded}

\begin{verbatim}
##   s1     s2 s3     s4     s5     s6    s7
## 1  1 0.9375  1 0.8125 0.7500 0.6250 0.500
## 1  0 0.0625  0 0.1875 0.2500 0.3750 0.500
## 2  0 0.0625  0 0.0625 0.1875 0.1250 0.125
## 2  1 0.9375  1 0.9375 0.8125 0.8750 0.875
## 3  0 0.0000  0 0.1250 0.0000 0.1875 0.125
## 3  1 1.0000  1 0.8750 1.0000 0.8125 0.875
## 4  0 0.0000  0 0.0000 0.0625 0.0625 0.250
## 4  1 1.0000  1 1.0000 0.9375 0.9375 0.750
\end{verbatim}

\begin{enumerate}
\def\labelenumi{\arabic{enumi}.}
\setcounter{enumi}{1}
\tightlist
\item
  Utilice la función barplot() y el argumento col = c (``black'',
  ``white'') en esta matriz. El gráfico que se obtiene ofrece una
  descripción general de la evolución de la Sensación de ardor con el
  tiempo.
\end{enumerate}

\begin{Shaded}
\begin{Highlighting}[]
\CommentTok{\# Generación de gráfico para evaluar la sensación de ardor con el tiempo}

\FunctionTok{par}\NormalTok{(}\AttributeTok{mar =} \FunctionTok{c}\NormalTok{(}\DecValTok{5}\NormalTok{, }\DecValTok{5}\NormalTok{, }\DecValTok{5}\NormalTok{, }\DecValTok{5}\NormalTok{))}
\NormalTok{original }\OtherTok{\textless{}{-}} \FunctionTok{barplot}\NormalTok{(data,}
        \AttributeTok{ylim=}\FunctionTok{c}\NormalTok{(}\DecValTok{0}\NormalTok{,}\DecValTok{4}\NormalTok{),}
        \AttributeTok{main=}\StringTok{"Evolución de la sensación de ardor con el tiempo"}\NormalTok{,}
        \AttributeTok{xlab=}\StringTok{\textquotesingle{}Semanas\textquotesingle{}}\NormalTok{,}
        \AttributeTok{ylab=}\StringTok{\textquotesingle{}Frecuencia relativa\textquotesingle{}}\NormalTok{,}
        \AttributeTok{col =} \FunctionTok{c}\NormalTok{ (}\StringTok{"black"}\NormalTok{, }\StringTok{"white"}\NormalTok{)}
\NormalTok{)}
\end{Highlighting}
\end{Shaded}

\includegraphics{iymprueba_files/figure-latex/unnamed-chunk-12-1.pdf}

\begin{enumerate}
\def\labelenumi{\arabic{enumi}.}
\setcounter{enumi}{2}
\tightlist
\item
  Cambie el gráfico anterior para que las barras que representan las
  frecuencias estén en rojo. Los números de las semanas deben estar en
  azul y en la parte superior del gráfico en lugar del fondo. Los
  números
\end{enumerate}

\begin{Shaded}
\begin{Highlighting}[]
\CommentTok{\# Generación de gráfico para evaluar la sensación de ardor con el tiempo con los cambios solicitados}

\CommentTok{\# Amplia el margen de la hoja del gráfico}
\FunctionTok{par}\NormalTok{(}\AttributeTok{mar =} \FunctionTok{c}\NormalTok{(}\DecValTok{5}\NormalTok{, }\DecValTok{5}\NormalTok{, }\DecValTok{5}\NormalTok{, }\DecValTok{5}\NormalTok{))}
\NormalTok{cambios }\OtherTok{\textless{}{-}} \FunctionTok{barplot}\NormalTok{(data,}
        \AttributeTok{ylim=}\FunctionTok{c}\NormalTok{(}\DecValTok{0}\NormalTok{,}\DecValTok{4}\NormalTok{),}
        \AttributeTok{xaxt =} \StringTok{\textquotesingle{}n\textquotesingle{}}\NormalTok{,}
        \AttributeTok{col =} \FunctionTok{c}\NormalTok{ (}\StringTok{"red"}\NormalTok{, }\StringTok{"white"}\NormalTok{),}
        \AttributeTok{col.axis =} \StringTok{"blue"}\NormalTok{,}
        \AttributeTok{font.axis =} \DecValTok{2}\NormalTok{,}

      
\NormalTok{)}

\FunctionTok{title}\NormalTok{(}\StringTok{"Evolución de la sensación de ardor con el tiempo"}\NormalTok{, }\AttributeTok{line =} \FloatTok{3.5}\NormalTok{)}
\CommentTok{\# Asigna los númeroas de las semanas a la parte superior del gráfico}
\FunctionTok{axis}\NormalTok{(}\DecValTok{3}\NormalTok{, }\AttributeTok{at =} \FunctionTok{seq}\NormalTok{(}\AttributeTok{from =} \FloatTok{0.7}\NormalTok{, }\AttributeTok{to =} \FloatTok{7.9}\NormalTok{,}\AttributeTok{by =} \FloatTok{1.2}\NormalTok{),}\AttributeTok{labels =} \FunctionTok{c}\NormalTok{(}\StringTok{"s1"}\NormalTok{, }\StringTok{"s2"}\NormalTok{, }\StringTok{"s3"}\NormalTok{, }\StringTok{"s4"}\NormalTok{,}\StringTok{"s5"}\NormalTok{, }\StringTok{"s6"}\NormalTok{, }\StringTok{"s7"}\NormalTok{), }
     \AttributeTok{col.axis =} \StringTok{"blue"}\NormalTok{, }\AttributeTok{font.axis =} \DecValTok{2}\NormalTok{, }\AttributeTok{tick =} \ConstantTok{FALSE}\NormalTok{)}
\end{Highlighting}
\end{Shaded}

\includegraphics{iymprueba_files/figure-latex/unnamed-chunk-13-1.pdf}

\end{document}
